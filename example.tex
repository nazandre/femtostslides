\documentclass{beamer}
%\documentclass[handout]{beamer}

\usepackage{femtostslides}
\usepackage[utf8]{inputenc}
\usepackage[english]{babel}

\title[Short title]{Long title}

\subtitle{Subtitle} % (optional)

\author{André Naz}
% - Use the \inst{?} command only if the authors have different
%   affiliation.

\institute{Femto-st}

%\date{2015}

\begin{document}

\begin{frame}
  \titlepage
\end{frame}

\section{First Section}

\subsection{First Subsection}

\begin{frame} \frametitle{FemtoSTSlides: FEMTO-ST Unofficial Latex-Beamer Theme}
  
  \begin{itemize}
  \item This is FemtoSTSlides: FEMTO-ST unofficial Latex-Beamer theme.
  \item It is based on ensislides from Mathieu Moy: \url{http://www-verimag.imag.fr/~moy/ensitools/beamer/}.
  \item It comes with many freeware icons from Icojam: \url{http://www.icojam.com/}. 
  \pause
  \question{Do you like it ?}
  \warning{This is not FEMTO-ST official theme!}
  \end{itemize}

\end{frame}

\begin{frame} \frametitle{Usage}
  
  \begin{itemize}
  \item Checkout the theme from GitHub: \url{http://www-verimag.imag.fr/~moy/ensitools/}.
  \item Compile this example using the command ``make''.
  \item Open example.pdf
  \end{itemize}
\end{frame}

\section{Last Section}

\begin{frame}
  \frametitle{Questions}
  \begin{center}
  \includegraphics[keepaspectratio=true,width=.45\paperwidth]{fig/question-2.png}\\
  Any question ?
  \end{center}
\end{frame}

\end{document}
